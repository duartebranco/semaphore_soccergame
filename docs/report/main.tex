%!TEX encoding = UTF-8 Unicode

\documentclass[code,math]{relatorio-deti}

\title{Projeto 2: Simulação de jogo de futebol}
\cadeira{Sistemas Operativos}
\relatorioAno{2024/2025}

\membro{Duarte Gabriel Castro Branco}{119253}

\usepackage{lmodern}
\usepackage{lipsum}
\usepackage{tocloft}
\usepackage{wasysym}

\begin{document}

\maketitle

\tableofcontents
\clearpage

%% \begin{listing}[H]
%% 	\centering
%% 	\begin{minted}{C}
%%         int ImageWidth(const Image img) {
%%           assert(img != NULL);
%%           return img->width;
%%         }
%%     \end{minted}
%% 	\caption{exemplo código}
%% 	\label{code}
%% \end{listing}

\setcounter{page}{1}

%-----NEW-CHAPTER------%
\chapter{Introdução}

O objetivo deste projeto é simular um jogo de futebol. O jogo é composto por 2 equipas, cada uma com os respetivos 4 jogadores e 1 guarda-redes. Existe também um único árbitro.
Existem jogadores a mais e cabe-nos a nós dar uso aos semáforos para tratarmos da concorrência e sincronização entre os jogadores, guarda-redes e árbitro.

Para isto, foi-nos pedido que completássemos os módulos:

\begin{itemize}
    \item \textbf{semSharedMemGoalie}
    \item \textbf{semSharedMemPlayer}
    \item \textbf{semSharedMemReferee}
\end{itemize}

Neste relatório, vou explicar o que foi feito em cada um destes módulos.

%-----NEW-CHAPTER------%
\chapter{semSharedMemGoalie}

Neste capítulo, vamos detalhar a implementação do módulo \textbf{semSharedMemGoalie}, que é responsável pelas operações realizadas pelo guarda-redes no jogo de futebol simulado. As operações/funções que nos foram pedidas para completar são:

\begin{itemize}
    \item \textbf{arriving}: O guarda-redes leva algum tempo a chegar e atualiza o seu estado.
    \item \textbf{goalieConstituteTeam}: O guarda-redes constitui uma equipa, caso seja possível
    \item \textbf{waitReferee}: O guarda-redes espera pelo árbitro para iniciar o jogo.
    \item \textbf{playUntilEnd}: O guarda-redes joga até o final do jogo.
\end{itemize}

%-----NEW-SECTION------%
\section{Função \textbf{arrive}}

A função \textbf{arrive} é responsável por simular a chegada do guarda-redes ao campo. Ela atualiza o estado do guarda-redes para \textbf{ARRIVING} e salva o estado interno. A função também faz o guarda-redes esperar durante um tempo aleatório antes de prosseguir. É de notar que esta função é necessária para a implementação de todos os outros módulos.

\begin{minted}{c}
static void arrive(int id)
{    
    if (semDown (semgid, sh->mutex) == -1)  {                          
        perror ("error on the up operation for semaphore access (GL)");
        exit (EXIT_FAILURE);
    }

    sh->fSt.st.goalieStat[id] = ARRIVING;
    saveState(nFic, &sh->fSt);

    if (semUp (semgid, sh->mutex) == -1) {                             
        perror ("error on the down operation for semaphore access (GL)");
        exit (EXIT_FAILURE);
    }

    usleep((200.0*random())/(RAND_MAX+1.0)+60.0);
}
\end{minted}

%-----NEW-SECTION------%
\section{Função \textbf{goalieConstituteTeam}}

A função \textbf{goalieConstituteTeam} verifica se há jogadores suficientes para formar uma equipa. Se houver, o guarda-redes forma a equipa e espera que os jogadores se registem. Caso contrário, o guarda-redes espera até que uma equipa possa ser formada. Existe também o caso em que o guarda-redes chega tarde e não consegue formar a equipa. A função retorna o id da equipa formada, ou 0 se o guarda-redes chegou tarde.
 
Comecei por incrementar o número de guarda-redes que chegaram e que estão livres, isto é necessário para verificar se há jogadores suficientes para formar equipa. Também inicializei uma nova variável booleana - \textit{waiting} - que vai ser necessária mais à frente.

\begin{minted}{c}
static int goalieConstituteTeam (int id)
{
    int ret = 0;
    // New variable
    bool waiting = false;
    
    if (semDown (semgid, sh->mutex) == -1)  {
        perror ("error on the up operation for semaphore access (GL)");
        exit (EXIT_FAILURE);
    }

    sh->fSt.goaliesArrived++;
    sh->fSt.goaliesFree++;      
\end{minted}

Depois trato dos guarda-redes que chegaram atrasados, ou seja, se o número de guarda-redes que chegaram é maior ou igual ao número total de guarda-redes, o estado do guarda-redes é atualizado para \textbf{LATE} e a função retorna 0, indicando que o guarda-redes chegou tarde.

\begin{minted}{c}
if (sh->fSt.goaliesArrived >= NUMGOALIES) {
    sh->fSt.st.goalieStat[id] = LATE;
    saveState(nFic, &sh->fSt);
    ret = 0;
\end{minted}

Se há jogadores e guarda-redes suficientes para formar uma equipa, o estado do guarda-redes é atualizado para \textbf{FORMING\_TEAM}. O número de jogadores e guarda-redes livres é decrementado e os jogadores são acordados para se registarem na equipa. Guarda-se o estado interno e incrementa-se o id da equipa. Por fim, o árbitro é avisado de que a equipa está formada.

\begin{minted}{c}
} else if (sh->fSt.playersFree >= NUMTEAMPLAYERS && 
           sh->fSt.goaliesFree >= NUMTEAMGOALIES) {
    sh->fSt.st.goalieStat[id] = FORMING_TEAM;
    saveState(nFic, &sh->fSt);

    sh->fSt.goaliesFree -= NUMTEAMGOALIES;
    sh->fSt.playersFree -= NUMTEAMPLAYERS;

    for (int i = 0; i < NUMTEAMPLAYERS; i++) {
        if (semUp (semgid, sh->playersWaitTeam) == -1) {                                                         
            perror ("error on the down operation for semaphore access (GL)");
            exit (EXIT_FAILURE);
        }    

        if (semDown (semgid, sh->playerRegistered) == -1)  {                                                     
            perror ("error on the up operation for semaphore access (GL)");
            exit (EXIT_FAILURE);
        }
    }

    ret = sh->fSt.teamId;
    sh->fSt.teamId++;

    if (semUp (semgid, sh->refereeWaitTeams) == -1) {                                                         
        perror ("error on the down operation for semaphore access (GL)");
        exit (EXIT_FAILURE);
    }      
\end{minted}

Caso ainda não haja jogadores suficientes para formar a equipa, o estado do guarda-redes é atualizado para \textbf{WAITING\_TEAM} e a variável \textit{waiting} é definida como \textit{true}.

\begin{minted}{c}
} else {
    sh->fSt.st.goalieStat[id] = WAITING_TEAM;
    saveState(nFic, &sh->fSt);
    waiting = true;
}
\end{minted}

Depois destas verificações, a função sai da região crítica (código do professor) e, logo de seguida, verifica-se se a variável \textit{waiting} é \textit{true}. Se for, o guarda-redes espera até que uma equipa possa ser formada. Quando a equipa é formada, o id da equipa é retornado. 

\begin{minted}{c}
if (waiting) {
    if (semDown (semgid, sh->goaliesWaitTeam) == -1)  {
        perror ("error on the up operation for semaphore access (GL)");
        exit (EXIT_FAILURE);
    }

    ret = sh->fSt.teamId;

    if (semUp (semgid, sh->playerRegistered) == -1) {
        perror ("error on the down operation for semaphore access (GL)");
        exit (EXIT_FAILURE);
    }
}
return ret;
\end{minted}


%-----NEW-SECTION------%
\section{Função \textbf{waitReferee}}

A função \textbf{waitReferee} faz o guarda-redes esperar pelo árbitro para iniciar o jogo. O estado do guarda-redes é atualizado para \textbf{WAITING\_START} e o estado interno é salvo. Os últimos 2 semáforos são usados para sincronizar o guarda-redes com o árbitro.

\begin{minted}{c}
static void waitReferee (int id, int team)
{
    if (semDown (semgid, sh->mutex) == -1)  {                                                     
        perror ("error on the up operation for semaphore access (GL)");
        exit (EXIT_FAILURE);
    }

    sh->fSt.st.goalieStat[id] = (team == 1) ? WAITING_START_1 : WAITING_START_2;
    saveState(nFic, &sh->fSt);

    if (semUp (semgid, sh->mutex) == -1) {                                                         
        perror ("error on the down operation for semaphore access (GL)");
        exit (EXIT_FAILURE);
    }

    if (semDown (semgid, sh->playersWaitReferee) == -1)  {                                                     
        perror ("error on the up operation for semaphore access (GL)");
        exit (EXIT_FAILURE);
    }

    if (semUp (semgid, sh->playing) == -1) {                                                         
        perror ("error on the down operation for semaphore access (GL)");
        exit (EXIT_FAILURE);
    }
}
\end{minted}

%-----NEW-SECTION------%
\section{Função \textbf{playUntilEnd}}

A função \textbf{playUntilEnd} faz o guarda-redes jogar até o final do jogo. O estado do guarda-redes é atualizado para \textbf{PLAYING} e o estado interno é salvo. O último semáforo é para esperar pelo fim do jogo.

\begin{minted}{c}
static void playUntilEnd (int id, int team)
{
    if (semDown (semgid, sh->mutex) == -1)  {                                                     
        perror ("error on the up operation for semaphore access (GL)");
        exit (EXIT_FAILURE);
    }

    sh->fSt.st.goalieStat[id] = (team == 1) ? PLAYING_1 : PLAYING_2;
    saveState(nFic, &sh->fSt);

    if (semUp (semgid, sh->mutex) == -1) {                                                         
        perror ("error on the down operation for semaphore access (GL)");
        exit (EXIT_FAILURE);
    }

    if (semDown (semgid, sh->playersWaitEnd) == -1)  {
        perror ("error on the up operation for semaphore access (GL)");
        exit (EXIT_FAILURE);
    }
}
\end{minted}

%-----NEW-CHAPTER------%
\chapter{semSharedMemPlayer}

Neste capítulo, vamos detalhar a implementação do módulo \textbf{semSharedMemPlayer}, que é responsável pelas operações realizadas pelos jogadores no jogo de futebol simulado. As operações principais incluem:

\begin{itemize}
    \item \textbf{arrive}: O jogador leva algum tempo a chegar e atualiza seu estado.
    \item \textbf{playerConstituteTeam}: O jogador constitui uma equipa, caso seja possível
    \item \textbf{waitReferee}: O jogador espera pelo árbitro para iniciar o jogo.
    \item \textbf{playUntilEnd}: O jogador joga até o final do jogo.
\end{itemize}

\section{Função \textbf{arrive}}

A função \textbf{arrive} é responsável por simular a chegada do jogador ao campo. Ela atualiza o estado do jogador para \textbf{ARRIVING} e salva o estado interno, tal como no módulo \textbf{semSharedMemGoalie}.

\begin{minted}{c}
static void arrive(int id)
{
    if (semDown (semgid, sh->mutex) == -1) {                                                     
        perror ("error on the up operation for semaphore access (PL)");
        exit (EXIT_FAILURE);
    }

    sh->fSt.st.playerStat[id] = ARRIVING;
    saveState(nFic, &sh->fSt);

    if (semUp (semgid, sh->mutex) == -1) {                                                       
        perror ("error on the up operation for semaphore access (PL)");
        exit (EXIT_FAILURE);
    }

    usleep((100.0*random())/(RAND_MAX+1.0)+10.0);
}
\end{minted}

\section{Função \textbf{playerConstituteTeam}}

A função \textbf{playerConstituteTeam} verifica se há jogadores suficientes para formar uma equipa. Segue a mesma forma que a função \textbf{goalieConstituteTeam} do módulo anterior, embora com algumas diferenças.

Comecei por incrementar o número de jogadores que chegaram e que estão livres e inicializei uma nova variável, tal como fiz anteriormente.

\begin{minted}{c}
static int playerConstituteTeam (int id)
{
    int ret = 0;
    // New variable
    bool waiting = false;

    if (semDown (semgid, sh->mutex) == -1) {
        perror ("error on the up operation for semaphore access (PL)");
        exit (EXIT_FAILURE);
    }

    sh->fSt.playersArrived++;
    sh->fSt.playersFree++;      
\end{minted}

Se o número de jogadores que chegaram é maior ou igual ao número total de jogadores possíveis (2 equipas * 4 jogadores = 8), o estado do jogador é atualizado para \textbf{LATE} e a função retorna 0, indicando que o jogador chegou tarde.

\begin{minted}{c}
if (sh->fSt.playersArrived > (2 * NUMTEAMPLAYERS)) {
    sh->fSt.st.playerStat[id] = LATE;
    saveState(nFic, &sh->fSt);
    ret = 0;
\end{minted}

Se há jogadores e guarda-redes suficientes para formar uma equipa, o estado do jogador é atualizado para \textbf{FORMING\_TEAM}. O número de jogadores e guarda-redes livres é decrementado e os jogadores são acordados para se registarem na equipa. Os novos semáforos são utilizados para acordar os guarda-redes que estão à espera de formar uma equipa e para esperar que os jogadores confirmem o seu registo na equipa, sendo este processo repetido para todos os jogadores necessários. Guarda-se o estado interno e incrementa-se o id da equipa. Por fim, o árbitro é avisado de que a equipa está formada.

\begin{minted}{c}
} else if (sh->fSt.playersFree >= NUMTEAMPLAYERS && 
           sh->fSt.goaliesFree >= NUMTEAMGOALIES) {
    sh->fSt.st.playerStat[id] = FORMING_TEAM;
    saveState(nFic, &sh->fSt);

    sh->fSt.playersFree -= NUMTEAMPLAYERS;
    sh->fSt.goaliesFree -= NUMTEAMGOALIES;

    if (semUp (semgid, sh->goaliesWaitTeam) == -1) {                                                         
        perror ("error on the down operation for semaphore access (PL)");
        exit (EXIT_FAILURE);
    }
    if (semDown (semgid, sh->playerRegistered) == -1)  {                                                     
        perror ("error on the up operation for semaphore access (PL)");
        exit (EXIT_FAILURE);
    }
    for (int i = 0; i < (NUMTEAMPLAYERS - 1); i++) {
        // Wake up players
        if (semUp (semgid, sh->playersWaitTeam) == -1) {                                                     
            perror ("error on the down operation for semaphore access (PL)");
            exit (EXIT_FAILURE);
        }
        // Wait for players to acknowledge registration
        if (semDown (semgid, sh->playerRegistered) == -1)  {                                                 
            perror ("error on the up operation for semaphore access (PL)");
            exit (EXIT_FAILURE);
        }
    }
    
    ret = sh->fSt.teamId;
    sh->fSt.teamId++;

    if (semUp (semgid, sh->refereeWaitTeams) == -1) {                                                         
        perror ("error on the down operation for semaphore access (PL)");
        exit (EXIT_FAILURE);
    }      
\end{minted}

Caso ainda não haja jogadores suficientes para formar a equipa, o estado do jogador é atualizado para \textbf{WAITING\_TEAM} e a variável \textit{waiting} é definida como \textit{true}.

\begin{minted}{c}
} else {
    sh->fSt.st.playerStat[id] = WAITING_TEAM;
    saveState(nFic, &sh->fSt);
    waiting = true;
}
\end{minted}

Depois destas verificações, a função sai da região crítica (código do professor) e, logo em seguida, verifica-se se a variável \textit{waiting} é \textit{true}. Se for, o jogador espera até que uma equipa possa ser formada. Quando a equipa é formada, o id da equipa é retornado.

\begin{minted}{c}
if (waiting) {
    if (semDown (semgid, sh->playersWaitTeam) == -1) {                                                     
        perror ("error on the up operation for semaphore access (PL)");
        exit (EXIT_FAILURE);
    }

    ret = sh->fSt.teamId;

    if (semUp (semgid, sh->playerRegistered) == -1) {                                                         
        perror ("error on the down operation for semaphore access (PL)");
        exit (EXIT_FAILURE);
    }
}
return ret;
\end{minted}

\section{Função \textbf{waitReferee}}

A função \textbf{waitReferee} faz o jogador esperar pelo árbitro para iniciar o jogo. O estado do jogador é atualizado para \textbf{WAITING\_START} e o estado interno é salvo, pelo que não muda comparativamente com o módulo anterior.

\begin{minted}{c}
static void waitReferee (int id, int team)
{
    if (semDown (semgid, sh->mutex) == -1) {                                                     
        perror ("error on the up operation for semaphore access (PL)");
        exit (EXIT_FAILURE);
    }

    sh->fSt.st.playerStat[id] = (team == 1) ? WAITING_START_1 : WAITING_START_2;
    saveState(nFic, &sh->fSt);

    if (semUp (semgid, sh->mutex) == -1) {                                                         
        perror ("error on the down operation for semaphore access (PL)");
        exit (EXIT_FAILURE);
    }

    if (semDown (semgid, sh->playersWaitReferee) == -1) {                                                     
        perror ("error on the up operation for semaphore access (PL)");
        exit (EXIT_FAILURE);
    }

    if (semUp (semgid, sh->playing) == -1) {                                                         
        perror ("error on the down operation for semaphore access (PL)");
        exit (EXIT_FAILURE);
    }
}
\end{minted}

\section{Função \textbf{playUntilEnd}}

A função \textbf{playUntilEnd} faz o jogador jogar até o final do jogo. O estado do jogador é atualizado para \textbf{PLAYING} e o estado interno é salvo e, tal como a função anterior, mantém-se fiel à implementação anterior.

\begin{minted}{c}
static void playUntilEnd (int id, int team)
{
    if (semDown (semgid, sh->mutex) == -1) {                                                     
        perror ("error on the up operation for semaphore access (PL)");
        exit (EXIT_FAILURE);
    }

    sh->fSt.st.playerStat[id] = (team == 1) ? PLAYING_1 : PLAYING_2;
    saveState(nFic, &sh->fSt);

    if (semUp (semgid, sh->mutex) == -1) {                                                         
        perror ("error on the down operation for semaphore access (PL)");
        exit (EXIT_FAILURE);
    }

    if (semDown (semgid, sh->playersWaitEnd) == -1) {                                                     
        perror ("error on the up operation for semaphore access (PL)");
        exit (EXIT_FAILURE);
    }
}
\end{minted}

%-----NEW-CHAPTER------%
\chapter{semSharedMemReferee}

Por fim, vou detalhar a implementação do módulo \textbf{semSharedMemReferee}, que é responsável pelas operações realizadas pelo árbitro no jogo de futebol. As operações/funções principais incluem:

\begin{itemize}
    \item \textbf{arrive}: O árbitro leva algum tempo a chegar e atualiza o seu estado.
    \item \textbf{waitForTeams}: O árbitro espera que as equipas sejam formadas.
    \item \textbf{startGame}: O árbitro inicia o jogo.
    \item \textbf{play}: O árbitro permite que o jogo continue por um tempo.
    \item \textbf{endGame}: O árbitro termina o jogo.
\end{itemize}

%-----NEW-SECTION------%
\section{Função \textbf{arrive}}

A função \textbf{arrive} é responsável por simular a chegada do árbitro ao campo. Ela atualiza o estado do árbitro para \textbf{ARRIVING} e salva o estado interno, pelo que permanece praticamente igual aqui como nos outros módulos. 

\begin{minted}{c}
static void arrive ()
{
    if (semDown (semgid, sh->mutex) == -1) {                                                      /* enter critical region */
        perror ("error on the up operation for semaphore access (RF)");
        exit (EXIT_FAILURE);
    }

    sh->fSt.st.refereeStat = ARRIVING;
    saveState(nFic , &sh->fSt);

    if (semUp (semgid, sh->mutex) == -1) {                                                        /* leave critical region */
        perror ("error on the up operation for semaphore access (RF)");
        exit (EXIT_FAILURE);
    }
    
    usleep((100.0*random())/(RAND_MAX+1.0)+10.0);
}
\end{minted}

%-----NEW-SECTION------%
\section{Função \textbf{waitForTeams}}

A função \textbf{waitForTeams} faz o árbitro esperar até que as duas equipas estejam completamente formadas. O estado do árbitro é atualizado para \textbf{WAITING\_TEAMS} e o estado interno é salvo.

\begin{minted}{c}
static void waitForTeams ()
{
    if (semDown (semgid, sh->mutex) == -1) {                                                      /* enter critical region */
        perror ("error on the up operation for semaphore access (RF)");
        exit (EXIT_FAILURE);
    }

    sh->fSt.st.refereeStat = WAITING_TEAMS;
    saveState(nFic, &sh->fSt);

    if (semUp (semgid, sh->mutex) == -1) {                                                        /* leave critical region */
        perror ("error on the up operation for semaphore access (RF)");
        exit (EXIT_FAILURE);
    }

    // For each team
    for (int i = 1; i <= 2; i++) {
        if (semDown (semgid, sh->refereeWaitTeams) == -1) {
            perror ("error on the up operation for semaphore access (RF)");
            exit (EXIT_FAILURE);
        }
    }
}
\end{minted}

%-----NEW-SECTION------%
\section{Função \textbf{startGame}}

A função \textbf{startGame} faz o árbitro iniciar o jogo. O estado do árbitro é atualizado para \textbf{STARTING\_GAME} e o estado interno é salvo. Em seguida, o árbitro sinaliza para os jogadores e guarda-redes começarem a jogar.

\begin{minted}{c}
static void startGame ()
{
    if (semDown (semgid, sh->mutex) == -1) {                                                      /* enter critical region */
        perror ("error on the up operation for semaphore access (RF)");
        exit (EXIT_FAILURE);
    }

    sh->fSt.st.refereeStat = STARTING_GAME;
    saveState(nFic, &sh->fSt);    

    if (semUp (semgid, sh->mutex) == -1) {                                                        /* leave critical region */
        perror ("error on the up operation for semaphore access (RF)");
        exit (EXIT_FAILURE);
    }

    // Signal teams to start playing (10 players including the late ones)
    for (int i = 0; i < NUMPLAYERS; i++) {
        if (semUp (semgid, sh->playersWaitReferee) == -1) {
            perror ("error on the up operation for semaphore access (RF)");
            exit (EXIT_FAILURE);
        }
    }

    for (int i = 0; i < NUMPLAYERS; i++) {
        if (semDown (semgid, sh->playing) == -1) {
            perror ("error on the down operation for semaphore access (RF)");
            exit (EXIT_FAILURE);
        }
    }
}
\end{minted}

%-----NEW-SECTION------%
\section{Função \textbf{play}}

A função \textbf{play} faz com que o árbitro permita que o jogo continue por um tempo. O estado do árbitro é atualizado para \textbf{REFEREEING} e o estado interno é salvo. A função também faz o árbitro esperar durante um tempo aleatório antes de prosseguir.

\begin{minted}{c}
static void play ()
{
    if (semDown (semgid, sh->mutex) == -1) {                                                      /* enter critical region */
        perror ("error on the up operation for semaphore access (RF)");
        exit (EXIT_FAILURE);
    }

    sh->fSt.st.refereeStat = REFEREEING;
    saveState(nFic, &sh->fSt);

    if (semUp (semgid, sh->mutex) == -1) {                                                        /* leave critical region */
        perror ("error on the up operation for semaphore access (RF)");
        exit (EXIT_FAILURE);
    }

    usleep((100.0*random())/(RAND_MAX+1.0)+900.0);
}
\end{minted}

%-----NEW-SECTION------%
\section{Função \textbf{endGame}}

A função \textbf{endGame} faz com que o árbitro termine o jogo. O estado do árbitro é atualizado para \textbf{ENDING\_GAME} e o estado interno é salvo. Em seguida, o árbitro sinaliza para os jogadores e guarda-redes que o jogo terminou.

\begin{minted}{c}
static void endGame ()
{
    if (semDown (semgid, sh->mutex) == -1) {                                                      /* enter critical region */
        perror ("error on the up operation for semaphore access (RF)");
        exit (EXIT_FAILURE);
    }
    sh->fSt.st.refereeStat = ENDING_GAME;
    saveState(nFic , &sh->fSt);
    if (semUp (semgid, sh->mutex) == -1) {                                                        /* leave critical region */
        perror ("error on the up operation for semaphore access (RF)");
        exit (EXIT_FAILURE);
    }
    // Signal teams that the game has ended
    for (int i = 0; i < NUMPLAYERS; i++) {
        if (semUp (semgid, sh->playersWaitEnd) == -1) {
            perror ("error on the up operation for semaphore access (RF)");
            exit (EXIT_FAILURE);
        }
    }
}
\end{minted}

%-----NEW-CHAPTER------%
\chapter{Testes}

Para testar o projeto, é preciso compilar usando "\textit{make all}" dentro da pasta \textit{src}, depois, vamos para a pasta \textit{run} e podemos correr o programa resultante \textbf{probSemSharedMemSoccerGame}. Existem também mais 2 \textit{scripts} que ajudam nos testes, estes sendo: \textit{filter.sh} e \textit{run.sh}. O \textit{run.sh} permite correr o programa resultante várias vezes, o que ajuda a detetar certos erros e notar o comportamento "comum" do projeto. O \textit{filter.sh} permite uma melhor visualização do programa principal, sinalizando quando há mudanças de estados, sendo este o programa que mais usei no desenvolvimento deste projeto. É de notar que, na etapa de compilação, podemos também testar apenas um dos nossos módulos, usando "\textit{make pl}, \textit{gl} ou \textit{rf}", o que foi importante no desenvolvimento deste projeto.

Executei o \textit{filter.sh} e obtive:

\begin{minted}{c}
                     SoccerGame - Description of the internal state

 P00  P01  P02  P03  P04  P05  P06  P07  P08  P09   G00  G01  G02   R01 
   A    A    A    A    A    A    A    A    A    A     A    A    A     A 
   .    .    .    .    .    .    .    .    .    .     .    .    .     . 
   .    .    .    .    .    .    .    .    .    .     W    .    .     . 
   .    .    .    .    .    .    .    .    .    .     .    .    W     . 
   .    .    .    .    .    .    .    .    .    .     .    L    .     . 
   .    .    .    .    .    .    .    W    .    .     .    .    .     . 
   .    W    .    .    .    .    .    .    .    .     .    .    .     . 
   .    .    .    .    .    W    .    .    .    .     .    .    .     . 
   .    .    .    .    F    .    .    .    .    .     .    .    .     . 
   .    .    .    .    .    .    .    .    .    .     .    .    .     . 
   .    .    .    .    .    .    .    .    .    .     s    .    .     . 
   .    .    .    .    .    .    .    s    .    .     .    .    .     . 
   .    s    .    .    .    .    .    .    .    .     .    .    .     . 
   .    .    .    .    .    s    .    .    .    .     .    .    .     . 
   .    .    .    .    .    .    .    .    .    W     .    .    .     . 
   .    .    .    .    s    .    .    .    .    .     .    .    .     . 
   .    .    .    .    .    .    .    .    W    .     .    .    .     . 
   .    .    .    W    .    .    .    .    .    .     .    .    .     . 
   .    .    .    .    .    .    .    .    .    .     .    .    .     . 
   .    .    .    .    .    .    .    .    .    .     .    .    .     . 
   .    .    .    .    .    .    F    .    .    .     .    .    .     . 
   .    .    .    .    .    .    .    .    .    .     .    .    S     . 
   .    .    .    .    .    .    .    .    .    S     .    .    .     . 
   .    .    .    .    .    .    .    .    S    .     .    .    .     . 
   .    .    .    S    .    .    .    .    .    .     .    .    .     . 
   .    .    .    .    .    .    S    .    .    .     .    .    .     . 
   .    .    L    .    .    .    .    .    .    .     .    .    .     . 
   L    .    .    .    .    .    .    .    .    .     .    .    .     . 
   .    .    .    .    .    .    .    .    .    .     .    .    .     W 
   .    .    .    .    .    .    .    .    .    .     .    .    .     S 
   .    .    .    .    .    .    .    p    .    .     .    .    .     . 
   .    .    .    .    .    .    .    .    .    .     p    .    .     . 
   .    p    .    .    .    .    .    .    .    .     .    .    .     . 
   .    .    .    .    .    p    .    .    .    .     .    .    .     . 
   .    .    .    .    .    .    .    .    .    .     .    .    P     . 
   .    .    .    .    p    .    .    .    .    .     .    .    .     . 
   .    .    .    .    .    .    .    .    .    P     .    .    .     . 
   .    .    .    P    .    .    .    .    .    .     .    .    .     . 
   .    .    .    .    .    .    .    .    P    .     .    .    .     . 
   .    .    .    .    .    .    P    .    .    .     .    .    .     . 
   .    .    .    .    .    .    .    .    .    .     .    .    .     R 
   .    .    .    .    .    .    .    .    .    .     .    .    .     E 
\end{minted}

Os resultados obtidos foram os esperados. O goalie1 e os players0 e 2, como chegam em último, não conseguem formar equipa e ficam LATE. A cada 4 jogadores que chegam, começam a formar equipa (F - FORMING\_TEAM) e, assim que a equipa está formada, esperam pelo arbitro para começar o jogo (s - WAITING\_START\_1; S - WAITING\_START\_2). O árbitro chega e espera pelas equipas (W - WAITING\_TEAMS) e a seguir começa o jogo (S - STARTING\_GAME). Os jogadores e guarda-redes jogam (P - PLAYING\_1; P - PLAYING\_2) e o árbitro arbitra (R - REFEREEING). No final, o árbitro termina o jogo (E - ENDING\_GAME).

Também corri o \textit{script} run.sh, que correu o programa \textbf{probSemSharedMemSoccerGame} 1000 vezes. Fiz isto não para ver o resultado de cada iteração do \textit{script}, mas para verificar que o resultado final do projeto permite rodar uma grande quantidade de testes e que não existe a possibilidade de existir um conflito mais raro entre os semáforos que impossibilite o término do \textit{script} - \textit{deadlock}. Felizmente, o \textit{script} concluiu sem qualquer problema. 

Os testes realizados demonstraram que a implementação dos módulos \textbf{semSharedMemGoalie}, \textbf{semSharedMemPlayer} e \textbf{semSharedMemReferee} está correta e funcional. A sincronização baseada em semáforos e memória partilhada foi eficaz na coordenação das ações entre jogadores, guarda-redes e árbitro.
%-------------------------------------------%
\end{document}